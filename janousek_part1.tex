\section*{Proposing a Research Topic in the Mechanics of Soft Materials}

I am interested in investigating the viscoelastic behavior of tire tread rubber and its contribution to rolling resistance, with a potential focus on the role of surface roughness. In passenger vehicles, a 60\% increase in the coefficient of rolling resistance can result in an increase in fuel consumption of 12-18\% (\cite{swieczko-zurekTyreRollingResistance2017}). Additionally, through my participation in the MRacing FSAE team, I come in contact with tires very often and engage in discussions on how to model them and simulate them daily. While many existing models approximate rubber as hyperelastic or rely on empirical correlations, a constitutive modeling approach that explicitly incorporates viscoelasticity and surface roughness would potentially enable us to better predict the performance of a tire under different conditions.

The proposed project advances the current state of knowledge by integrating viscoelastic constitutive modeling with surface roughness effects to predict rolling resistance. The current state-of-the-art in tire modeling is dominated by empirical and semi-empirical approaches. The most prominent of these approaches is the "Magic Formula" developed by Hans Pacejka (\cite{pacejkaMagicFormula1992})). Pacejka's model is now a cornerstone of vehicle-dynamics simulations as it offers promising fits to tire force and moment data. The problem is, however, that it is essentially a black-box curve-fitting approach, without direct connections to the material-level constitutive physics such as viscoelastic hysteresis. As a result of that, it is not capable of providing insight into how changes in the rubber formulation in the tire will affect rolling resistance.

This course and its emphasis on continuum mechanics and constitutive modeling frameworks offers a way to go beyond these empirical fits. Classical models in viscoelasticity, such as the Maxwell and Kelvin-Voigt models, describe the time-dependent response in linear regimes. More advanced models, such as the Bergström-Boyce (\cite{bergstrom-boyceConstModelLargeStrain1998}) account for non-linear viscoelasticity and large strains. These are essential for modeling tire behavior. Additionally, Persson (\cite{perssonRubberFrictionContactMech2001}) and Klüppel and Heinrich (\cite{kluppel-heinrichRubberFrictionSelfAffineRoad2000}) have investigated how surface roughness and conditions can be coupled with viscoelastic response in order to predict energy dissipation.

By building on these studies, my ambition is to bridge the gap between rubber viscoelasticity and applied tire dynamics. Specifically, I will aim to implement a viscoelastic constitutive model to simulate hysteresis losses in cyclic deformation and then extend this analysis to surface roughness. This approach will make use of the concepts taught in this course: kinematics and deformation, conservation of energy with dissipation, and viscoelastic constitutive laws. 

Reducing rolling resistance is of significant societal and industrial importance. For internal combustion vehicles, it has the potential to lower fuel consumption and associated carbon emissions, while for electric vehicles it could lead to directly increasing the driving range and thus reducing the energy demand. This research could be used to contribute to the development of sustainable, energy-efficient tires. Beyond automotive applications, the modeling framework could also be used in other areas where viscoelastic contact and surface roughness are critical, such as in seals, vibration isolators, and soft robotics. In general, however, this project will aim to demonstrate the importance of connecting fundamental material physics and models to real-world engineering challenges, which could lead to a reduction of reliance on expensive experimental data collection.